\subsection{Parser}
\label{subsec:parser}

%%brief parser description
We manually inspected the documents provided by \ac{CLEF} in order to understand their \emph{structure} and be able to extract the body and ID of each document from them, which is shown in Listing ~\ref{lst:docformat}.

\begin{lstlisting}[label={lst:docformat},caption={Document format},captionpos=b, xleftmargin=.4\textwidth]
<DOC>
<DOCNO> ... </DOCNO>
<DOCID> ... </DOCID>
<TEXT>
 ... 
</TEXT>
</DOC>
\end{lstlisting}

In order to do that, we created a tool called \emph{parser} that has been essential for extracting information from documents in the specified format 
used by \ac{TREC}. The parser helps us create structured objects that are used for analysis and indexing within the \ac{IR} system.

Here are the key components we implemented:
\begin{itemize}
    \item  \textbf{\texttt{ParsedDocument}}: represents a parsed document to be indexed.  It has two attributes: \texttt{ID} for the unique identifier of the document and \texttt{body} for the document's content. This class provides functionalities to set and retrieve documents' attributes.
    \\
    \item  \textbf{\texttt{DocumentParser}}: represents an abstract class providing basic functionalities to iterate over the elements of a \texttt{ParsedDocument}, reading and parsing its content.

    \\
    \item  \textbf{\texttt{LongEvalDocumentParser}}: specific \texttt{DocumentParser} for the LongEval corpus. It provides an implementation of a parser for the documents in the TREC format. The parser reads a document and returns a \texttt{ParsedDocument} that contains the \texttt{ID} and the \texttt{body} of the document.
\end{itemize}   

We used  the \texttt{LongEvalDocumentParser} and the \texttt{ParsedDocument} in the indexer to represent the content of the documents that are 
in the directory specified by \texttt{docsDir}. The first one has been used to iterate over the content of the specified document, while the second one has been used to represent a document to be indexed.