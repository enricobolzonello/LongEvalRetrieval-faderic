\subsubsection{SpellChecker}
\label{subsubsec:spellchecker}

In the searcher, we tried to implement a spell checker for the query tokens.
For every query token, using the \texttt{SpellChecker} class of Lucene \citep{Lucene}, the spell checker analyzes the token and, based on a dictionary of words stored in the resources folder, suggests one similar (not by meaning but by \emph{spelling}) word.
This new word is appended at the end of the query, after the original token, and is given a weight of 0,2 to the new word. This is done because if we match the original token, it's more relevant than matching the new modified token.
We do this only for those queries that are composed of only one word (token), because, otherwise, the performance dropped.
In fact, for these queries, suggesting one similar but different word can make a difference in case the one word inserted in the query is \emph{misspelled}.
In the case of queries composed of more than one token, it's difficult that all the words are misspelled, so by changing all the words, the performances obviously drop.
