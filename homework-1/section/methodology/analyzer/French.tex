\subsubsection{French analyzer}

The \texttt{FrenchLongEvalAnalyzer} component is in charge of processing French language texts, it is composed by:
\begin{itemize}
    \item \textbf{Tokenizing}: the \texttt{StandardTokenizer} is used, which exploits the Word Break rules from Unicode Text Annex \#29 \citep{UAX29};
    \item \textbf{Character folding}: the \texttt{ICUFoldingFilter} is used, which applies the foldings from Unicode Technical Report \#30 \citep{UTR30}. This is useful to fold upper cases, accents and other kinds of complex characters;
    \item \textbf{Elision removal}: the \texttt{ElisionFilter} is used, which removes the elision from words;
    \item \textbf{Stopword removal}: the \texttt{StopFilter} is used, which removes the given stopwords from the tokens. In this system we have tried using the default Lucene \citep{Lucene} stoplist and custom one, generated by picking the 50 most frequent terms in the documents;
    \item \textbf{Position filtering}: a custom \texttt{TokenFilter} has been implemented to set the \texttt{positionIncrement} attribute of all tokens to a specific value. This will be useful to ignore the \texttt{positionIncrement} due to removed stopwords in the search phase when we will use the proximity between tokens, as explained in Subsection~\ref{subsubsec:shingles};
    \item \textbf{Stemming}: this process is useful to reduce words to their base form, in this system we have tried using the Snowball French \citep{FrSnowball} and Light \citep{LightStem} stemmers.
\end{itemize}

In Table~\ref{tab:french-analyzer} we show an example of the analyzing process for the French language.

\begin{table}[b]
    \caption{French analyzer process}
    \label{tab:french-analyzer}
    \centering
    \begin{tabular}{|c|c|}
        \toprule
        \textbf{Step} & \textbf{Tokens}\\
        \midrule
          & La méthode d'analyse de texte est\\ 
          & essentielle pour l'extraction d'informations.\\ 
         \midrule
         Tokenizing & [La, méthode, d'analyse, de, texte, est,\\
         & essentielle, pour, l'extraction, d'informations]\\
         \midrule
         Character folding & [la, methode, d'analyse, de, texte, est,\\
          & essentielle, pour, l'extraction, d'informations]\\
         \midrule
         Stopword removal & [methode, analyse, texte,\\
         (50 most freq.) & essentielle, extraction, informations]\\
         \midrule
         Stemming & [method, analys, text,\\
         (Ligth) & esentiel, extraction, inform]\\
        \bottomrule
    \end{tabular}
\end{table}