\subsection{Indexer}
\label{subsec:indexer}

Indexing is a crucial step where we create a searchable database, called \emph{index}, for the parsed documents. The index contains important information about the documents, 
such as the words and phrases they contain, their frequency, and their location within the document. 
Indexing allows us to store the documents in a \emph{structured} manner, which greatly speeds up the retrieval process by enabling users to search for documents based on keywords or phrases. To achieve this, we developed the following components:

\begin{itemize}
    \item \textbf{\texttt{DirectoryIndexer}}: indexes the documents located in a specified directory and its sub-directories. It accepts various parameters, including the directory containing the documents to be indexed, the \texttt{DocumentParser}, the \texttt{Analyzer}, the \texttt{Similarity} to be used for indexing, the expected number of documents and the location where the index will be stored. Our code ensures that the document directory is readable and the index directory is writable before initiating the indexing process. Additionally, it keeps track of statistics, such as the number of indexed files and documents. \\
The main component of the class is the \texttt{index} method, which is in charge of performing the actual indexing of the documents. This method iterates through the documents in the directory, extracting their content and adding it to the index. During all the iterations, some statistics indexing is given, such as the time taken every 10 thousand documents. Finally, the index is closed.
    \item \textbf{\texttt{BodyField}}: represents the body field of a document. This field has the following properties:

\begin{itemize}
	\item it is \emph{tokenized}, meaning that the body is broken into words, or tokens, to make the search more accurate and flexible;
	\item \emph{frequencies} and also the \emph{positions} of the tokens are stored, in order to allow for phrase queries with proximity, as explained in Section~\ref{subsubsec:shingles};
	\item the body content is \emph{stored}, even if this had an impact on the index size, this was needed in the search phase in order to pass documents bodies to the reranker, as explained in Section~\ref{subsubsec:reranker}.
\end{itemize}

\end{itemize}

\begin{table}[tbp]
    \caption{Indexing performances}
    \label{tab:index-perf}
    \centering
    \begin{tabular}{|c|>{\centering\arraybackslash}p{0.1\linewidth}|c|c|c|>{\centering\arraybackslash}p{0.1\linewidth}|c|}
        \toprule
        \textbf{Collection} & \textbf{Docs size} (GB) & \textbf{Stoplist} & \textbf{Stemmer} & \textbf{Body terms} & \textbf{Index size} (GB) & \textbf{Time} (s)  \\
        \midrule
        French & 7.99 & Default & Snowball & 7,497,875 & 6.98 & 1224\\
        & & 50 most freq. & Light & 7,459,058 & 6.95 & 842\\
        \midrule
        English & 7.27 & Default & Snowball & 7,253,947 & 6.49 & 1041 \\
        & & 50 most freq. & Krovetz & 7,451,647 & 6.43 & 848 \\
        \bottomrule
    \end{tabular}
\end{table}

In Table~\ref{tab:index-perf} are reported the index performances obtained using the analyzers described in Section~\ref{subsec:analyzer} and the setup described in Section~\ref{sec:setup}.